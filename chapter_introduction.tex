\xchapter{Introduction}{ }

\section{History of State Foundation Family health}

According to \cite{Silva:2010}, in the Brazil the \ac{HCS} began in 1960, however, its operation was regulate by the Unified Health System in the 90's from the act N. 8.080 of 19 September 1990, which regulates throughout the national territory the actions and health services carried out on a permanent or occasional basis by individuals or legal entities governed by public or private law. According to information from the website of the State Foundation for Family Health, the State Foundation for \ac{SFFH} started in Bahia in 2012.    

The \ac{SFFH} is a public non-profit organization operating in 69 municipalities in the State of Bahia, provided a series of health care services such as family health, humanized childbirth and, part of the services provided by the foundation. Being officially instituted in 2009.
ON April 16, 2012, \ac{SFFH} assumed the management of the home hospitalization program, offering the home care service, free of residents of the city of Salvador and metropolitan areas.
 
The foundation currently operates in 9 bases, serving patients hospitalized at home, with a team of professionals composed by two physicians, one nurse, four nursing technicians, and one physiotherapist, also counting on support professionals: social worker, one nutritionist and one speech therapist.
One of the bases of the \ac{SFFH} is found in the State General Hospital in the city of Salvador, serving several patients hospitalized at home.

%pesquisa Operacional
\section{Operational Research}

Operational Research is an applied mathematics branch that makes use of mathematical, statistical, and algorithmic models to aid decision making. 

According to \citeonline{hamdy:2008} The first formal activities involving Operational Research began in England during World War II when a team of British scientists decided to make scientifically-based decisions on the best use of war material.
After the war, problems in this field were adapted to improve efficiency and productivity in various industries.

An Operational Research model is composed of hard and soft constraints that must be satisfied to minimize or maximize an objective function.  The  Operational Research model is viable if satisfy all constraints, and is optimal if it finds the best value \cite{hamdy:2008}.

According to \citeonline{hamdy:2008}, the most used technique for solving Operational Research problems is \ac{LIP}. However, there are several techniques for solving Operational Research problems, such as Integer Programming, Dynamic Programming, Constraint Programming and non-linear methods.

To find the solution to the problem is defined algorithms that provide calculation rules that are applied repeatedly to the problem. In each repetition, the solution gets closer to becoming optimal. There are mathematical models so complex that it is impossible to solve them by available optimization algorithms, in which case heuristics are used\cite{hamdy:2008}.

 Problem Definition, where the scope of the problem under investigation is defined. Model Construction, where the translation of the definition of the problem into mathematical relations takes place. Model Solution, where the optimization algorithms are applied. Model Validation, where is verified if the proposed model works as expected. Implementation of the solution, where results are translated into operational instructions so that it can be used by people or another program \cite{hamdy:2008}.

%roteamento de veículo
\section{Routing and Scheduling Problems}

The \ac{VRPTW} is a combinatorial optimization problem which has an objective determine a route of minimal cost. That problem was defined by \citeonline{desrochers:1992} as a generalization of \ac{VRP} whose a service has a Time Window that defines the upper and lowers bound to beginning the service.

According to \cite{laporte:1992}, the \ac{VRP} can be described as the problem of designing optimal delivery or collection routes from one or several depots to a number of geographically scattered cities or customers, subject to constraints.

%escalonamento de equipes
The Nurse Scheduling Problem is a NP-Hard operational research problem whose objective is to find an optimal way of assigning a set of available nurses to required shifts. These problem include hard constraints that should not be violated, and soft constraints. \cite{santos:2015}

\section{Home Care Service}
%Serviço de atenção domiciliar
The \ac{HCS} is a category of health care, composed by a set of activities of prevention, rehabilitation, and treatment of diseases provided at home. That kind of service has become increasingly present, offering an alternative form for people with stable clinical conditions which requiring specialized care.

The \ac{HCS} ensure more comfort to the patients, facilitate family support, reduce risks of hospital-acquired infection and reduce hospital occupancy.  
However, the \ac{HCS} have some challenges to the health professional and to domicile manager. The necessity of the displacement of the health professional, all patients are taken care of within the provided time, the scheduling of the work of health professionals involved in the home care, the increasing costs for the family or manager of domicile when  is necessary maintain electrical equipment connected to the maintenance of patient, and the possible stay of the caregiver.

Due to the intricate planning of work schedules of professionals and the difficulties encountered completing all daily service in the expected time without exceeding the workload of health professionals, the \ac{HCS} has aroused the interest of several researchers. The problem studied was defined as \ac{HHCRSP}. 

The \ac{HHCRSP} approach an integrated way the Vehicle Routing Problem and Nurse Scheduling Problem, with the aim of developing a work schedule for the \ac{HCS}, in order to each caregiver, visit a set of patients, pause and finish your activities within work time window \cite{trabelsi:2012}. 
To satisfy the \ac{HHCRSP} constraints is necessary to define defining the work scale of the professionals and routes that will be followed by SAD vehicles, since the design of the routes depends on some factors, such as the need for some patient.

According to \cite{Kergosien:2009} the \ac{HHCRSP} is reducible to Multiple Traveling Salesman Problem, classifying it into the problem class NP-Hard. Therefore, it is unlikely that will not be possible to determine a polynomial algorithm to solve it.
According to \cite{cheng:98}, \cite{bachouch:2010},\cite{tozlu:2016} and \cite{cattafi:2012}, the \ac{HHCRSP} is solving manually in several countries, which can lead to unsatisfactory results.
Finally, is estimated in \cite{holm:2014} with professionals involved on \ac{HCS} spend between $18 \%$ and $26\%$ on their work day in the vehicle by making transfers between the service points, which reinforces the need to use optimization techniques.

Due to the nature of the problem, were found various methods to determine a good solution to \ac{HHCRSP}. To investigate in a formal way what methods have been developed to solve it, to find possible research problems, to ensure the applicability of this review and provide support to other researchers of the subject at the time of obtaining study materials, we make a \ac{SLR}.
No other \ac{SLR} was found in the researches, but other literature reviews, classified by \cite{Kitchenham:2007} as ad-hoc review, were published by \cite{fikar:2017} and \cite{mohamed:2017}.

%organização do trabalho