\xchapter{Routing Scheduling Problems}{ }

\section{Staff Scheduling Problem}

The Staff Scheduling Problem is an assignment problem whose set of tasks or shifts is associated with a set of people according to their abilities so that the organization can satisfy the demand for its goods or services respecting the work restrictions and preference of the staff members.

According to \citeonline{sier:2004} is difficult to determine optimal solutions that minimize costs, meet employee preferences, distribute shifts equitably among employees and satisfy all the workplace constraints.

The Staff Scheduling Problem has a set of soft and hard constraints, in which the hard constraints can't be violated and a penalty function $f_r$ is associated to each soft constraints $r$. If the soft constraint is perfectly compiled, the penalty function $f_r = 0$, otherwise, $f_r$ increase according to number of violations \cite{blochiger:2003}.

The Staff Scheduling Problem was defined by \citeonline{blochiger:2003} as follow:

Be $S = \{s_1,s_ 2, \ldots, s_{|S|}\}$ a set of staff members and $T = \{t_1, t_2, \ldots, t_{|T|}\}$ a set of tasks. A solution of the Staff Scheduling Problem can be represented by a matrix $M$, in which each cell $m_{t_is_j}$ contains the value referring to assignment of task $t_i \in T$ to staff member $s_j \in S$. as in Table~\ref{time_table_block}

\begin{table}[h]
\centering
\caption{Example of allocation to a set of tasks to a set of staff members. \label{time_table_block}} 
\begin{tabular}{r|l|l|l|l}
      & $s_1$ & $s_1$ & $s_3$ & $s_4$ \\ \hline
$t_1$ & $0$   & $0$   & $1$   & $0$  \\ \hline
$t_2$ & $1$   & $0$   & $0$   & $0$  \\ \hline
$t_3$ & $0$   & $0$   & $0$   & $1$  \\ \hline
$t_4$ & $0$   & $1$   & $0$   & $0$ 
\end{tabular}
\end{table}


\section{Traveling Salesman Problem}

The \ac{TSP} is a Combinatorial Optimization Problem whose given a set of locals and the distances between each local, the problem has as objective find a minimum path who visit each city exactly once \citeonline{goyal:2010}.

\citeonline{laporte:1992} defined the \ac{TSP} as a graph $G = (V,A)$ in which $V$ is a set of vertices, $A$ is a set of arcs and $\tau:V\times V \rightarrow \mathds{R}$  a function who associate each pair of vertices to distance.  

The  \ac{TSP} was defined mathematically by \citeonline{laporte:1992} as follow:

\begin{equation}\label{objectiveFunction}
\mbox{minimize} \sum{c_{ij}x_{ij}}
\end{equation}

subject to: \\

\begin{equation}\label{degreeConstraints}
 \sum{x_{ij}} = 1, i = \{1, \ldots, n\}, j = \{1, \ldots, n\}
\end{equation}

\begin{equation}\label{eliminationConstraints}
 \sum{x_{ij}} \leqslant S - 1, S \subset V, 2 \leqslant S \leqslant n - 2
\end{equation}

\begin{equation}
x_{ij} \in {0,1},
i,j = 1, \ldots, n, 
i \neq j
\end{equation}

Where

$x_{ij}$ Is a decision variable equal to 1 if the arc (i,j) is used in solution,
The equation~\ref{objectiveFunction} is an objective function that describes the minimum path,
the constraint~\ref{degreeConstraints} specify that every vertex is visited exactly once and
the constraint~\ref{eliminationConstraints} prohibit the formation of loops.

Several variations of the \ac{TSP} have been studied. In this work will be described the \ac{MTSP} and the \ac{TSPTW}. 

The \acl{MTSP} is a generalization of the \ac{TSP} in which it considers a set of staff members and aims to determine for each staff member, a minimum cost route in which each city should be visited once by a single staff member~\cite{meng:2012}.

In the \acl{TSPTW}, a client is assigned a time window in which the upper and lower limit is defined for the beginning of an activity and aims to find a minimum cost circuit starting and ending at the same location, visiting a set of clients only once respecting the visiting time window \cite{urrutia:2010}.

\section{Vehicle Routing Problem}

The \ac{VRP} was defined by \citeonline{gilbert:1992} as a problem which consists in generate a minimum cost route in which each local is visited once by in which each place is visited only once, by a vehicle that part and return from the same place, the depot.

The \ac{VRP} may be represented as a graph $G = (V,A)$ in which $V = \{v_0,v_1,\ldots,v_{|V|} \}$ is a set of vertices that represents the locals as the vertice $v_0$ represent the depot, $A = \{(i,j) : i\in V, j \in V, i \neq j \}$ is a set of arcs that represent the path of $i$ and $j$, and the associated a cost $c_{ij}$ which can represents travel cost.

Be  $Q = \{ q_1, q_2, \ldots, q_{|Q|} \}$ a set of customers, $x_{ij}$  a variable which is equal to 1 when a arc is in a solution. The \ac{VRP} was described by \citeonline{laporte:2007} as:

\begin{equation}\label{5}
Minimize \sum_{[i,j] \in A} c_{ij}x_{ij}
\end{equation}
Subject \ to:
\begin{equation}\label{6}
\sum_{j \in V-\{v_0\}} x_{0j} = 2m
\end{equation}
\begin{equation}\label{7}
\sum_{i<k} x_{ik} + \sum_{j>k} x_{kj} = 2 (k \in V-\{v_0\})
\end{equation}
\begin{equation}\label{8}
\sum_{i \in Q, j \notin S or i \notin Q, j \in S } x_{ij} \geqslant 2b(Q) (Q \subset V-\{v_0\})
\end{equation}
\begin{equation}\label{9}
x_{ij} = 0 or 1 (i,j \in V-{v_0})
\end{equation}
\begin{equation}\label{10}
x_{0j} = 0, 1 or 2 (j \in V-{v_0})
\end{equation}

Where

The objective function represented by~\ref{5} is minimize the total cost of the route. The constraint~\ref{6} define the degree of vertice  $v_0$ being that right side of the equation can be a constant if $m$ is known previously, or a variable if $m$ isn't known previously, the constraint~\ref{7} ensures that one edge arrives and another part of each edge and the constraint~\ref{8} is a lower bound of vehicles needed to attend each customer.

\subsection{Vehicle Routing Problem With Time Window}

The \ac{VRPTW} is a combinatorial optimization problem which has an objective determine a route of minimal cost. That problem was defined by \citeonline{desrochers:1992} as a generalization of \ac{VRP} whose a service has a Time Window that defines the upper and lowers bound to beginning the service.


The \ac{VRPTW} was represented by \citeonline{desrochers:1992} as a graph $G = (V,A)$ in which $V = \{v_0,v_1,\ldots,v_{|V|} \}$ is a set of vertices that represents the locals as the vertice $v_0$ represent the depot, and $A = \{(i,j) : i\in V, j \in V, i \neq j \}$ is a set of arcs. Associated to each arc $(i,j) \in A$ exist a cost $c_{ij}$ that represent travel cost. A variable $t_{i,j}$ representing a total time of service. A time window defined by $[e_i,l_i]$ in which $e_i$ is a lower bound and  $l_i$ is a upper bound to beginning the visit.

Be $R = \{ r_1, r_2, \ldots, r_{|R|} \}$ a set of routes and $\delta_{ir}$ a constant with value $1$ if exists customers to be visited on the route and 0 otherwise. The \ac{VRPTW} was formulated mathematically by \citeonline{desrochers:1992} as:

\begin{equation}\label{11}
min \sum_{r \in R} c_rx_r
\end{equation}
subject \ to:
\begin{equation}\label{12}
\sum_{r \in R} \delta_{ir}x_r = 1, i \in V-{v_0}
\end{equation}
\begin{equation}\label{13}
x_r \in \{ 0,1 \}, r \in R
\end{equation}

Where

The equation \ref{11} represents the objective function of \ac{VRPTW}, defining that the sum of the costs of the arcs of a route must be minimal. The restriction \ref{12} defines that it is $ x_r $, a binary variable with value $1$ if the route $r$ is used and $0$ otherwise, and the decision variable \ref{13} defines the values that should be used.